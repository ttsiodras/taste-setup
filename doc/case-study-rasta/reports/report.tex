\documentclass[11pt]{article}

\usepackage{graphicx}
\usepackage{color}
\usepackage[utf8]{inputenc}
\usepackage{url}
\usepackage{xspace}
\usepackage{minitoc}
\usepackage{listings}
\usepackage{epsfig}
\usepackage{hyperref}
\usepackage{aeguill}

\renewcommand{\rmdefault}{ppl}
\renewcommand{\bfdefault}{b}
\usepackage[scaled]{helvet}
\renewcommand{\ttdefault}{pcr}
\linespread{1.05}
\normalfont
\usepackage[T1]{fontenc}


\renewcommand*\thesection{\thechapter\arabic{section}}

\lstset{language=bash,frame=single,basicstyle=\footnotesize}

\lstdefinelanguage{aadl}
{morekeywords={aadlboolean,aadlinteger,aadlreal,aadlstring,access,all,and,
        annex,applies,binding,bus,calls,classifier,connections,constant,
        data,delta,device,end,enumeration,event,extends,false,features,flow,
        flows,group,implementation,in,inherit,initial,inverse,is,list,memory,
        mode,modes,none,not,of,or,out,package,parameter,path,port,private,
        process,processor,properties,property,provides,public,range,
        reference,refined,refines,requires,server,set,sink,source,
        subcomponents,subprogram,system,thread,to,true,type,units,value},
morecomment=[l]{--}}

% Layout for listings

\lstset{language=aadl,
        basicstyle=\scriptsize\sffamily,
        aboveskip=.1cm, % \smallskipamount, % \bigskipamount,
        belowskip=.1cm, % \smallskipamount, % \bigskipamount,
        abovecaptionskip=.1cm, % \smallskipamount, % \medskipamount,
        belowcaptionskip=.1cm, % \smallskipamount, % \bigskipamount,
        xleftmargin=.0cm,
        captionpos=b,
        tabsize=3}


\lstdefinelanguage{real}
{morekeywords={cardinal,is\_subcomponent\_of,is\_bound\_to,sum,foreach,theorem,check,end,
               memory\_set, processor\_set,min,max,
               process\_set,is\_provided\_class,is\_connected\_to,
               system\_set,thread\_set,in,bus\_set,data\_set,is\_accessed\_by,connection\_set,
               virtual\_processor\_set,Cardinal,virtual\_bus\_set,property\_exists},
morecomment=[l]{--}}


% Layout for listings

\lstset{language=aadl,
        basicstyle=\scriptsize\sffamily,
        aboveskip=.1cm, % \smallskipamount, % \bigskipamount,
        belowskip=.1cm, % \smallskipamount, % \bigskipamount,
        abovecaptionskip=.1cm, % \smallskipamount, % \medskipamount,
        belowcaptionskip=.1cm, % \smallskipamount, % \bigskipamount,
        xleftmargin=.0cm,
        captionpos=b,
        tabsize=3}


\newcommand{\bi}{\begin{itemize}}
\newcommand{\ei}{\end{itemize}}

\newcommand{\be}{\begin{enumerate}}
\newcommand{\ee}{\end{enumerate}}

\newcommand{\tbf}[1]{\textbf{#1}}
\newcommand{\tit}[1]{\textit{#1}}
\newcommand{\ttt}[1]{\texttt{#1}}
\newcommand{\tsc}[1]{\textsc{#1}}
\newcommand{\tsl}[1]{\textsl{#1}}



\newcommand{\Concept}[1]{#1\xspace}

\newcommand{\vmware}{\textsc{VMWare}\copyright\xspace}
\newcommand{\vmwareplayer}{\textsc{VMWare Player}\copyright\xspace}
\newcommand{\aadl}{\Concept{AADL}}
\newcommand{\ada}{\Concept{Ada}}
\newcommand{\arinc}{\Concept{ARINC653}}
\newcommand{\belllapadula}{\Concept{Bell-Lapadula}}
\newcommand{\biba}{\Concept{Biba}}
\newcommand{\taste}{\Concept{Taste}}
\newcommand{\cheddar}{\Concept{Cheddar}}
\newcommand{\chineesewall}{\Concept{Chineese Wall}}
\newcommand{\eclipse}{\Concept{Eclipse}}
\newcommand{\libpok}{\Concept{libpok}}
\newcommand{\marte}{\Concept{MARTE}}
\newcommand{\mils}{\Concept{MILS}}
\newcommand{\myccmhi}{\Concept{MyCCM-HI}}
\newcommand{\ocarina}{\Concept{Ocarina}}
\newcommand{\leon}{\Concept{LEON}}
\newcommand{\gnatforleon}{\Concept{Gnatforleon}}
\newcommand{\ocl}{\Concept{OCL}}
\newcommand{\osate}{\Concept{OSATE}}
\newcommand{\pok}{\Concept{POK}}
\newcommand{\posix}{\Concept{POSIX}}
\newcommand{\polyorbhi}{\Concept{PolyORB-HI}}
\newcommand{\powerpc}{\Concept{PowerPC}}
\newcommand{\qemu}{\Concept{QEMU}}
\newcommand{\rasta}{\Concept{RASTA}}
\newcommand{\real}{\Concept{REAL}}
\newcommand{\rtems}{\Concept{RTEMS}}
\newcommand{\sparc}{\Concept{Sparc}}
\newcommand{\spoq}{\Concept{SPOQ}}
\newcommand{\standardcc}{\Concept{Critères Communs}}
\newcommand{\standarddo}{\Concept{DO178B}}
\newcommand{\sysml}{\Concept{SysML}}
\newcommand{\uml}{\Concept{UML}}
\newcommand{\xml}{\Concept{XML}}
\newcommand{\xcov}{\Concept{xcov}}

\newcommand{\onehalffig}[3]{%
  \begin{figure}[htbp]
    \centerline{\epsfig{file=#1.pdf,width=.50\textwidth}}
    \caption{#2}
    \label{#3}
  \end{figure}
}

\newcommand{\onemedfig}[3]{%
  \begin{figure}[htbp]
    \centerline{\epsfig{file=#1.pdf,width=.75\textwidth}}
    \caption{#2}
    \label{#3}
  \end{figure}
}


\newcommand{\onefullfig}[3]{%
  \begin{figure}[htbp]
    \centerline{\epsfig{file=#1.pdf,width=.90\textwidth}}
    \caption{#2}
    \label{#3}
  \end{figure}
}

\newcommand{\onehugefig}[3]{%
  \begin{figure}[htbp]
    \centerline{\epsfig{file=#1.pdf,width=1\textwidth}}
    \caption{#2}
    \label{#3}
  \end{figure}
}

\newcommand{\hugeaadlmodel}[3]{%
  \begin{figure}[htbp]
    \centerline{\epsfig{angle=90,file=#1.pdf,width=15cm}}
    \caption{#2}
    \label{#3}
  \end{figure}
}



\newcommand{\onesmallfig}[3]{%
  \begin{figure}[htbp]
    \centerline{\epsfig{file=#1.pdf,width=.40\textwidth}}
    \caption{#2}
    \label{#3}
  \end{figure}
}


\newcommand{\onetinyfig}[3]{%
  \begin{figure}[htbp]
    \centerline{\epsfig{file=#1.pdf,width=.30\textwidth}}
    \caption{#2}
    \label{#3}
  \end{figure}
}



\title{
\centerline{\epsfig{file=imgs/assert-logo.pdf,width=.75\textwidth}}
RASTA case-study}

\author{~}

\begin{document}

\maketitle

\newpage

\tableofcontents

\newpage

\section{Overall architecture}
The overall architecture of the system is depicted in figure
\ref{figure:archi-globale}. Each LEON board is connected with a \rasta card
through the PCI bus. The RASTA provides connectors for Spacewire and serial
buses.

The two LEON boards communicate through the Spacewire bus and standalone Linux
computers are connected to the \rasta cards through the serial bus.

Finally, each LEON board is connected to an FPGA through the serial bus.

   \onefullfig{imgs/archi-globale}
   {Overall architecture of the distributed system}
   {figure:archi-globale}


\section{AADL architecture}


   \subsection{Overview}
   To test the integration of our architecture, we propose to design a
   distributed architecture with four nodes:
   \begin{enumerate}
      \item
         Two nodes run on a LEON board. Each one is connected to an FPGA through
         a serial bus. These nodes communicates through the RASTA board using
         the spacewire link.
      \item
         Two Linux computers that communication with the LEON through the serial
         interface of the RASTA boards.
   \end{enumerate}

      \subsubsection{Test application}
      To test the correctness of our design and the feasability of this
      implementation, we propose to make a simple ping/pong example :
      \begin{itemize}
         \item
            One LEON board \textit{pings} the other through the RASTA card and
            its spacewire link.
         \item
            Each Linux computer \textit{pings} one LEON board using the serial
            line of each RASTA card.
         \item
            Each LEON board executes a process that sends data to the FPGA
            through the serial line.
      \end{itemize}


   \subsection{Graphical representation}
   The figure \ref{figure:archi-aadl} depicts the \aadl architecture of our
   proposed case study. It does
   not provide all details (no thread, \ldots) but gives an overview of the different processors,
   processes, \ldots.

   \hugeaadlmodel{imgs/archi-aadl}
   {AADL model}
   {figure:archi-aadl}


   \newpage

   \subsection{Textual representation}

   \lstinputlisting[language=aadl,numberstyle=\tiny]{aadl/processors.aadl}
   \lstinputlisting[language=aadl,numberstyle=\tiny]{aadl/application.aadl}
   \lstinputlisting[language=aadl,numberstyle=\tiny]{aadl/types.aadl}
   \lstinputlisting[language=aadl,numberstyle=\tiny]{aadl/main.aadl}
   \lstinputlisting[language=aadl,numberstyle=\tiny]{aadl/devices.aadl}
   \lstinputlisting[language=aadl,numberstyle=\tiny]{aadl/buses.aadl}
   \lstinputlisting[language=aadl,numberstyle=\tiny]{aadl/protocols.aadl}

\end{document}

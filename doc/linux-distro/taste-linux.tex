\documentclass[11pt]{book}

\usepackage{graphicx}
\usepackage{color}
\usepackage[utf8]{inputenc}
\usepackage{url}
\usepackage{xspace}
\usepackage{minitoc}
\usepackage{listings}
\usepackage{epsfig}
\usepackage{fullpage}
\usepackage[pdftex, colorlinks=true,linkcolor=blue, urlcolor=blue, citecolor=blue]{hyperref}
\usepackage{hyperref}
\usepackage{rotating}
\usepackage{multirow}

\renewcommand{\rmdefault}{ppl}
\renewcommand{\bfdefault}{b}
\usepackage[scaled]{helvet}
\renewcommand{\ttdefault}{pcr}
\linespread{1.05}
\normalfont
\usepackage[T1]{fontenc}


%\renewcommand*\thesection{\thechapter\arabic{section}}

\lstset{language=bash,frame=single,basicstyle=\footnotesize}

\lstdefinelanguage{aadl}
{morekeywords={aadlboolean,aadlinteger,aadlreal,aadlstring,access,all,and,
        annex,applies,binding,bus,calls,classifier,connections,constant,
        data,delta,device,end,enumeration,event,extends,false,features,flow,
        flows,group,implementation,in,inherit,initial,inverse,is,list,memory,
        mode,modes,none,not,of,or,out,package,parameter,path,port,private,
        process,processor,properties,property,provides,public,range,
        reference,refined,refines,requires,server,set,sink,source,
        subcomponents,subprogram,system,thread,to,true,type,units,value},
morecomment=[l]{--}}

% Layout for listings

\lstset{language=aadl,
        basicstyle=\scriptsize\sffamily,
        aboveskip=.1cm, % \smallskipamount, % \bigskipamount,
        belowskip=.1cm, % \smallskipamount, % \bigskipamount,
        abovecaptionskip=.1cm, % \smallskipamount, % \medskipamount,
        belowcaptionskip=.1cm, % \smallskipamount, % \bigskipamount,
        xleftmargin=.0cm,
        captionpos=b,
        tabsize=3}


\lstdefinelanguage{real}
{morekeywords={cardinal,is\_subcomponent\_of,is\_bound\_to,sum,foreach,theorem,check,end,
               memory\_set, processor\_set,min,max,
               process\_set,is\_provided\_class,is\_connected\_to,
               system\_set,thread\_set,in,bus\_set,data\_set,is\_accessed\_by,connection\_set,
               virtual\_processor\_set,Cardinal,virtual\_bus\_set,property\_exists},
morecomment=[l]{--}}


% Layout for listings

\lstset{language=aadl,
        basicstyle=\scriptsize\sffamily,
        aboveskip=.1cm, % \smallskipamount, % \bigskipamount,
        belowskip=.1cm, % \smallskipamount, % \bigskipamount,
        abovecaptionskip=.1cm, % \smallskipamount, % \medskipamount,
        belowcaptionskip=.1cm, % \smallskipamount, % \bigskipamount,
        xleftmargin=.0cm,
        captionpos=b,
        tabsize=3}


\newcommand{\bi}{\begin{itemize}}
\newcommand{\ei}{\end{itemize}}

\newcommand{\be}{\begin{enumerate}}
\newcommand{\ee}{\end{enumerate}}

\newcommand{\tbf}[1]{\textbf{#1}}
\newcommand{\tit}[1]{\textit{#1}}
\newcommand{\ttt}[1]{\texttt{#1}}
\newcommand{\tsc}[1]{\textsc{#1}}
\newcommand{\tsl}[1]{\textsl{#1}}



\newcommand{\Concept}[1]{#1\xspace}

\newcommand{\vmware}{\textsc{VMWare}\footnotesize{\copyright}\xspace}
\newcommand{\adacore}{\textsc{Adacore}\footnotesize{\copyright}\xspace}
\newcommand{\vmwareplayer}{\textsc{VMWare Player}\footnotesize{\copyright}\xspace}
\newcommand{\aadl}{\Concept{AADL}}
\newcommand{\ada}{\Concept{Ada}}
\newcommand{\arinc}{\Concept{ARINC653}}
\newcommand{\debian}{\Concept{Debian}}
\newcommand{\belllapadula}{\Concept{Bell-Lapadula}}
\newcommand{\biba}{\Concept{Biba}}
\newcommand{\taste}{\Concept{Taste}}
\newcommand{\cheddar}{\Concept{Cheddar}}
\newcommand{\chineesewall}{\Concept{Chineese Wall}}
\newcommand{\eclipse}{\Concept{Eclipse}}
\newcommand{\libpok}{\Concept{libpok}}
\newcommand{\marte}{\Concept{MARTE}}
\newcommand{\mils}{\Concept{MILS}}
\newcommand{\myccmhi}{\Concept{MyCCM-HI}}
\newcommand{\ocarina}{\Concept{Ocarina}}
\newcommand{\leon}{\Concept{LEON}}
\newcommand{\gnatforleon}{\Concept{Gnatforleon}}
\newcommand{\ocl}{\Concept{OCL}}
\newcommand{\osate}{\Concept{OSATE}}
\newcommand{\pok}{\Concept{POK}}
\newcommand{\posix}{\Concept{POSIX}}
\newcommand{\powerpc}{\Concept{PowerPC}}
\newcommand{\qemu}{\Concept{QEMU}}
\newcommand{\real}{\Concept{REAL}}
\newcommand{\rtems}{\Concept{RTEMS}}
\newcommand{\sparc}{\Concept{Sparc}}
\newcommand{\spoq}{\Concept{SPOQ}}
\newcommand{\standardcc}{\Concept{Critères Communs}}
\newcommand{\standarddo}{\Concept{DO178B}}
\newcommand{\sysml}{\Concept{SysML}}
\newcommand{\uml}{\Concept{UML}}
\newcommand{\xml}{\Concept{XML}}
\newcommand{\xcov}{\Concept{xcov}}

\newcommand{\onehalffig}[3]{%
  \begin{figure}[htbp]
    \centerline{\epsfig{file=#1.pdf,width=.50\textwidth}}
    \caption{#2}
    \label{#3}
  \end{figure}
}

\newcommand{\onemedfig}[3]{%
  \begin{figure}[htbp]
    \centerline{\epsfig{file=#1.pdf,width=.75\textwidth}}
    \caption{#2}
    \label{#3}
  \end{figure}
}


\newcommand{\onefullfig}[3]{%
  \begin{figure}[htbp]
    \centerline{\epsfig{file=#1.pdf,width=.90\textwidth}}
    \caption{#2}
    \label{#3}
  \end{figure}
}

\newcommand{\onehugefig}[3]{%
  \begin{figure}[htbp]
    \centerline{\epsfig{file=#1.pdf,width=1\textwidth}}
    \caption{#2}
    \label{#3}
  \end{figure}
}


\newcommand{\onesmallfig}[3]{%
  \begin{figure}[htbp]
    \centerline{\epsfig{file=#1.pdf,width=.40\textwidth}}
    \caption{#2}
    \label{#3}
  \end{figure}
}


\newcommand{\onetinyfig}[3]{%
  \begin{figure}[htbp]
    \centerline{\epsfig{file=#1.pdf,width=.30\textwidth}}
    \caption{#2}
    \label{#3}
  \end{figure}
}

\newcommand{\boxfixme}[1]{%
\begin{center}
\fbox{%
   \begin{minipage}{0.5\textwidth}
   \centering{\textbf{!!! FIXME !!!}}
   \rule{\linewidth}{0.5pt}
   #1
   \end{minipage}%
}
\end{center}
}

\title{
\centerline{\epsfig{file=imgs/logo-taste.pdf,width=.75\textwidth}}
TASTE-Linux distribution documentation \\ v1.1}

\author{Julien Delange}
\begin{document}

\maketitle

\newpage

\tableofcontents

\newpage

\chapter{Introduction to the TASTE Linux distribution}
The TASTE Linux distribution was designed to be used for the implementation of
Robotics applications. However, it can be tailored for the development of other
real-time applications.

The distribution is based on Debian, stable version. On top of Debian, we build
a customized kernel based on the vanilla kernel (from
\url{http://www.kernel.org}) and the Xenomai patch.

   \section{Availability}
   The latest version of the distribution is available on the internet, at the
   following address:
   \url{http://download.tuxfamily.org/taste/linux-taste/linux-taste.img}


\chapter{Use the TASTE Linux distribution}

   \section{Running the distribution}
      \subsection{Deploy on an USB stick}
      You can put the distribution on a USB stick in order to be able to deploy
      it automatically on a system. To do so, download the image and issue the
      following command on Linux:
      \begin{verbatim}
      dd if=taste-linux.img of=/path/to/your/usb/key/device
      \end{verbatim}

      It will write the taste linux image on your usb stick so that you can run
      the distro from the usb stick. Be careful, this would erase all existing
      data on your usb stick and even destroy the existing filesystem
      (partitions and so on).


      \subsection{Emulation with QEMU}
      To run the image, you can use qemu for rapid
      development/prototyping.
      The following command line would start the distro in qemu :
      \begin{verbatim}
   qemu -hda taste-linux.img -boot c -net nic,vlan=0  \
      -net tap,vlan=0,ifname=tap0,script=/etc/qemu-ifup -m 256 -localtime
      \end{verbatim}

         \subsubsection{Increase QEMU speed} 
         You can run the system faster by using the virtualization
         functions of your architecture (if supported). In that case,
         use the -enable-kvm option when qemu is launched.
         Make sure that the following modules are loaded into your kernel
         (depending on your processor manufacturer): \texttt{kvm-intel} or
         \texttt{kvm-amd}).

         \subsubsection{Networking support within QEMU}
         If you want to connect your QEMU instance with your Linux system,
         you have to configure a bridge. On regular Debian/Linux system,
         you can do that be editing your network configuration. Proceed
         as it :
         \begin{enumerate}
            \item
            Edit \texttt{/etc/network/interfaces} and add the following lines 
            (assuming that your computer is connected using \texttt{eth0} and
            that your network connection is configured with dhcp):
            \begin{verbatim}
            auto br0
            iface br0 inet dhcp
            bridge_ports eth0
            bridge_fd 9
            bridge_hello 2
            bridge_maxage 12
            bridge_stp off
            \end{verbatim}

            \item
            Restart the network by ussing the following command:
            \begin{verbatim}
            /etc/init.d/networking restart
            \end{verbatim}
         
            \item
            Edit \texttt{/etc/qemu-ifup} and put the following lines:
            \begin{verbatim}
            #!/bin/sh
            echo "Executing /etc/qemu-ifup"
            echo "Bringing up $1 for bridged mode..."
            sudo /sbin/ifconfig $1 0.0.0.0 promisc up
            echo "Adding $1 to br0..."
            sudo /usr/sbin/brctl addif br0 $1
            sleep 2
            \end{verbatim}

            \item
            Launch QEMU with these additional parameters:
            \begin{verbatim}
   -net nic,vlan=0 -net tap,vlan=0,ifname=tap0,script=/etc/qemu-ifup
            \end{verbatim}

            So, the following command would start qemu with KVM
            support and network configured:

            \begin{verbatim}
   qemu -hda taste-linux.img -boot c \
   -net nic,vlan=0 -net tap,vlan=0,ifname=tap0,script=/etc/qemu-ifup \
   -m 256 -localtime -enable-kvm -clock rtc
            \end{verbatim}

            \item
            If you want to avoid to type your password each time
            the \texttt{/etc/qemu-ifup} script is invoked, you have to configure
            sudo. In that case, invoke \texttt{visudo} (as root) and change
            your configuration like this:

            \begin{verbatim}
   Cmnd_Alias      QEMU=/sbin/ifconfig,/sbin/modprobe,/usr/sbin/brctl
   youruser ALL=NOPASSWD: QEMU
            \end{verbatim}

            or

            \begin{verbatim}
   Cmnd_Alias      QEMU=/sbin/ifconfig,/sbin/modprobe,/usr/sbin/brctl
   %yourgroup ALL=NOPASSWD: QEMU
            \end{verbatim}
         \end{enumerate}

   \section{Accounts and password}
   The root password is:
   \begin{verbatim}
      taste
   \end{verbatim}

   The system also defines a regular user account:
   \begin{itemize}
      \item
         \textbf{Username:} taste
      \item
         \textbf{Password:} taste
   \end{itemize}

   \section{Directories and path}
   The following directories have to be considered if you plan to develop for
   this target/platform:
   \begin{itemize}
      \item
         Xenomai main directory: \texttt{/usr/xenomai}
      \item
         Xenomai libraries: \texttt{/usr/xenomai/lib}
   \end{itemize}


   \section{Configuration of the analogy devices}
   First of all, you have to load the appropriate kernel module (for example
   \texttt{analogy\_ni\_pcimio}). Then, you have to invoke the
   \texttt{analogy\_config} from Xenomai tools to be able to load and control
   the card. Invoke it like this:
   \begin{verbatim}
   analogy_config DEVICEID DEVICEDRIVER
   \end{verbatim}


   For example, you can invoke the tool as it:
   \begin{verbatim}
   analogy_config analogy0 analogy_ni_pcimio
   \end{verbatim}

   Then, in that case, the device would be available in the analogy framework in
   Xenomai under the name \texttt{analogy0}.

   \section{Kernel modules automatically loaded}
   When the system starts, the following modules are automatically loaded:
   \begin{itemize}
      \item
         \texttt{xeno\_posix}
      \item
         \texttt{xeno\_native}
   \end{itemize}

   \section{Kernel modules available}
   The kernel modules specific to Xenomai are located in the following
   directories:
   \begin{itemize}
      \item
         \texttt{/lib/modules/KERNEL\_VERSION/kernel/kernel/xenomai}
      \item
         \texttt{/lib/modules/KERNEL\_VERSION/kernel/drivers/xenomai}
   \end{itemize}

   The following modules are available:
   \begin{itemize}
      \item \texttt{xeno\_nucleus}
      \item \texttt{xeno\_uitron}
      \item \texttt{xeno\_vrtx}
      \item \texttt{xeno\_native}
      \item \texttt{xeno\_vxworks}
      \item \texttt{xeno\_rtdm}
      \item \texttt{xeno\_psos}
      \item \texttt{xeno\_posix}
      \item \texttt{xeno\_analogy}
      \item \texttt{analogy\_s526}
      \item \texttt{analogy\_parport}
      \item \texttt{analogy\_8255}
      \item \texttt{analogy\_ni\_mio}
      \item \texttt{analogy\_ni\_tio}
      \item \texttt{analogy\_ni\_pcimio}
      \item \texttt{analogy\_ni\_mite}
      \item \texttt{analogy\_fake}
      \item \texttt{analogy\_loop}
      \item \texttt{xeno\_rtipc}
      \item \texttt{xeno\_rtdmtest}
      \item \texttt{xeno\_klat}
      \item \texttt{xeno\_switchtest}
      \item \texttt{xeno\_irqbench}
      \item \texttt{xeno\_timerbench}
      \item \texttt{xeno\_sigtest}
      \item \texttt{xeno\_can\_peak\_pci}
      \item \texttt{xeno\_can\_mem}
      \item \texttt{xeno\_can\_esd\_pci}
      \item \texttt{xeno\_can\_ems\_pci}
      \item \texttt{xeno\_can\_peak\_dng}
      \item \texttt{xeno\_can\_sja1000}
      \item \texttt{xeno\_can\_plx\_pci}
      \item \texttt{xeno\_can\_isa}
      \item \texttt{xeno\_can\_ixxat\_pci}
      \item \texttt{xeno\_can\_virt}
      \item \texttt{xeno\_can}
      \item \texttt{xeno\_16550A}
   \end{itemize}

   \section{Configuration of the kernel}
   The kernel sources are available in \texttt{/usr/src/linux}. The sources of
   Xenomai are also available in the  \texttt{/usr/src} directory. The
   configuration file of the kernel is available in
   \texttt{/usr/src/linux/.config}.

   \section{Rebuild your own kernel}
   You can rebuild your own kernel by issuing the following command in the
   \texttt{/usr/src/linux} directory:

   \begin{verbatim}
   make-kpkg --initrd kernel_image
   \end{verbatim}


   \section{Communication with the machine}
   A ssh server is installed within the VM. File transfer operations and access
   to the machine can be done using ssh of any sftp-compliant program.



\chapter{Use the Analogy layer for interaction with acquisition data boards}
The TASTE linux distribution embedds the xenomai real-time kernel. Is also
includes an analogy layer which purpose is to interface with acquisition boards,
such as the one supported by the comedi drivers.

This analogy layer provides user functions to communicate with acquisition
boards in a uniform and smooth way. In the following, we explain where you can
find relevant documentation to use the analogy layer and provide an example of
the use of these boards.

   \section{Documentation of the analogy layer}
   User-level functions that interacts with the analogy layer of Xenomai are
   documented in the Xenomai API document. It is available on the Xenomai
   website (\url{http://www.xenomai.org}), either in PDF or HTML
   format\footnote{you can
   check out this page for example
   \url{http://www.xenomai.org/documentation/xenomai-2.5/html/api/index.html}
   and search within the \texttt{Analogy API} module}.


   \section{Use the board: the \texttt{analogy\_config} tool}
   First of all, you must define the boards that will be used by your system.
   You do that using the \texttt{analogy\_config} tool. This tool specify which
   driver (kernel module) is used for each board. Each board is identified
   using a number, starting from 0 (so, the first board will be
   \texttt{analogy0}).

   So, to associate an analogy device with a driver, you just have to invoke the
   following command:
\begin{verbatim}
analogy_config analogy_device kernel_module
\end{verbatim}

   For example:
\begin{verbatim}
analogy_config analogy0 analogy_ni_pcimio
\end{verbatim}

   In this case, the card \texttt{analogy0} is identified as using the kernel
   module \texttt{analogy\_ni\_pcimio}. When using the analogy API, you will
   refer to it using the name \texttt{analogy0}.

   \section{Use the board and the analogy API}
   First of all, you have to include the header file \texttt{analogy.h} provided
   by the analogy layer of xenomai. For that, you have to use the following
   pre-processing code:

\begin{lstlisting}
   #include <xenomai/analogy/analogy.h>
\end{lstlisting}

   Then, you have to open the device. This is done by using the function
   \texttt{a4l\_open} function which opens the device and stores relevant
   information about it in a structure with the type \texttt{a4l\_desc\_t}. The
   following listing illustrates how to use this function (note that a return
   code different from 0 means that an error was raised when opening the device) :
   \begin{lstlisting}
   a4l_desc_t arm_device;
   ret = a4l_open (&arm_device, "analogy0");

   if (ret != 0)
   {
      printf ("[EXOARM] Error while opening the arm device, return code=%d \n", ret);
      return;
   }
   \end{lstlisting}

   Then, once the device is opened, you need to fill its descriptor
   informations. This is done by calling the \texttt{a4l\_fill\_desc} function,
   like this:
   \begin{lstlisting}
   ret = a4l_fill_desc (&arm_device);

   if (ret != 0)
   {
      printf ("[EXOARM] Error while calling fill_desc(), return code=%d \n", ret);
      return;
   }
   \end{lstlisting}


   \subsection{Getting channel information}
   Then, for each channel of the acquisition board, you need to get channel
   information. This is done with the \texttt{a4l\_get\_chinfo} function. For
   example, the following function call will retrieve information for the first
   channel of the device opened in the \texttt{arm\_device} variable. Note that
   a return code different from 0 means that an error happened.

   \begin{lstlisting}
   a4l_chinfo_t channel_infos;

   ret = a4l_get_chinfo (&arm_device, 0, 0, &channel_infos);
   if (ret != 0)
   {
      printf ("[EXOARM] Error invoking a4l_get_chinfo\n");
   }
   \end{lstlisting}


   \subsection{Getting range information}
   Then, for each channel, you need to get the range information, which indicate
   to maximum and minimum value that can be acquired on a particular channel.
   You can get this information by calling the \texttt{a4l\_get\_rnginfo}
   function. It will store information about data range in a
   \texttt{a4l\_rnginfo\_t} structure, that will be used later when
   acquiring/converting data.

   The following listing shows how to get range informations for the first
   channel on the device opened using the \texttt{arm\_device} variable. 

   \begin{lstlisting}
   a4l_rnginfo_t range_infos;
   ret = a4l_get_rnginfo (&arm_device, 0, 0, 0, &range_infos);
   if (ret != 0)
   {
      printf ("[EXOARM] Error invoking a4l_get_rnginfo\n");
   }
   \end{lstlisting}


   \subsection{Acquiring data}
   Then, you acquire the data using the \texttt{a4l\_sync\_read} function. This
   function gets \textit{raw} data, meaning that the acquired data cannot be
   used directly and has to be converted later on in a programming-language
   dependent type (for example, an \texttt{int} or a \texttt{double}).

   So, when you acquire a data, you just have to provide a buffer to store the
   data and specify its size. In the following listing, we acquire data on the
   first channel and store it in the buffer \texttt{raw}, which has a length of
   128 bytes.

   \begin{lstlisting}
   uint8_t raw[128];
   ret = a4l_sync_read (&arm_device, 0, 0, 0, &raw, 128);
   if (ret <= 0)
   {
       printf("[arm] Error while acquiring the data\n");
   }
   \end{lstlisting}

   \subsection{Convert RAW data to programming types}
   Finally, you need to convert the RAW data acquired previously to a type that
   can be used in your programming language. This is done with the serie of
   functions \texttt{a4l\_rawto*}. To convert the \textit{raw} data into a
   \texttt{double} type, you can use the \texttt{a4l\_rawtod} function.

   Conversion function requires the channel and range informations that has been
   taken previously (see the previous sections). The following listing
   illustrates how you can convert the \textit{raw} data acquired previously
   into a \texttt{double} type using the channel and range informations obtained
   previously in the variable \texttt{range\_infos} and \texttt{channe\_infos}.
   \begin{lstlisting}
   double converted_value;
   ret = a4l_rawtod (&channel_infos, &range_infos, &converted_value, &raw, 1);

   if (ret <= 0)
   {
       printf("[arm] Error while converting the data\n");
   }
   \end{lstlisting}


   \subsection{Initialization and acquisition example}
   The following listing shows an example of the use of an acquisition board. We
   assume the board was initialized using the \texttt{analogy\_config} and
   opened as \texttt{analogy0}. Then, it opens the data, get range and channel
   informations and finally acquires data on the first channel every second. It
   finally outputs the data on the standard output.


   \begin{lstlisting}
   #include <stdio.h>
   #include <xenomai/analogy/analogy.h>

   int main ()
   {
      int ret;
      a4l_desc_t      arm_device;

      a4l_chinfo_t   channel_infos;
      a4l_rnginfo_t  range_infos;

      double         val;
      uint8_t        raw[128];


      ret = a4l_open (&arm_device, "analogy0");

      if (ret != 0)
      {
         printf ("[EXOARM] Error while opening the arm device, return code=%d \n", ret);
         return;
      }
      arm_device.sbdata = malloc(arm_device.sbsize);


      ret = a4l_fill_desc (&arm_device);

      if (ret != 0)
      {
         printf ("[EXOARM] Error while calling fill_desc(), return code=%d \n", ret);
         return;
      }


      ret = a4l_get_chinfo (&arm_device, 0, 0, &channel_infos);
      if (ret != 0)
      {
         printf ("[EXOARM] Error invoking a4l_get_chinfo\n");
      }

      ret = a4l_get_rnginfo (&arm_device, 0, 0, 0, &range_infos);
      if (ret != 0)
      {
         printf ("[EXOARM] Error invoking a4l_get_rnginfo\n");
      }

      while (1)
      {
         ret = a4l_sync_read (&arm_device, 0, 0, 0, &raw, 128);
         if (ret <= 0)
         {
             printf("[arm] Error while acquiring the data\n");
         }

         ret = a4l_rawtod (channel_infos, range_infos, &val, &raw, 1);

         if (ret <= 0)
         {
             printf("[arm] Error while converting the data\n");
         }
      }
   }
   \end{lstlisting}

\chapter{Development for the TASTE Linux distribution}
   \section{Available software}
   The following development tools are available within the distribution:
   \begin{itemize}
      \item
         binutils
      \item
         GCC
      \item
         Xenomai with \texttt{POSIX} and \texttt{Native} skins
   \end{itemize}

\chapter{Annexes}

   \section{Links and websites}
   \begin{itemize}
      \item
         Debian - \url{http://www.debian.org}
      \item
         Xenomai - \url{http://www.xenomai.org}
      \item
         Comedi - \url{http://www.comedi.org}
      \item
         System configuration for QEMU and network concerns: \url{http://compsoc.dur.ac.uk/~djw/qemu.html}
   \end{itemize}


\bibliographystyle{unsrt}
\bibliography{biblio}

\nocite{*}

\end{document}

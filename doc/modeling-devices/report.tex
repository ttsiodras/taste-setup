\documentclass[11pt]{article}

\usepackage{graphicx}
\usepackage{color}
\usepackage[utf8]{inputenc}
\usepackage{url}
\usepackage{xspace}
\usepackage{minitoc}
\usepackage{listings}
\usepackage{epsfig}
\usepackage{hyperref}
\usepackage{aeguill}

\renewcommand{\rmdefault}{ppl}
\renewcommand{\bfdefault}{b}
\usepackage[scaled]{helvet}
\renewcommand{\ttdefault}{pcr}
\linespread{1.05}
\normalfont
\usepackage[T1]{fontenc}


\renewcommand*\thesection{\thechapter\arabic{section}}

\lstset{language=bash,frame=single,basicstyle=\footnotesize}

\lstdefinelanguage{aadl}
{morekeywords={aadlboolean,aadlinteger,aadlreal,aadlstring,access,all,and,
        annex,applies,binding,bus,calls,classifier,connections,constant,
        data,delta,device,end,enumeration,event,extends,false,features,flow,
        flows,group,implementation,in,inherit,initial,inverse,is,list,memory,
        mode,modes,none,not,of,or,out,package,parameter,path,port,private,
        process,processor,properties,property,provides,public,range,
        reference,refined,refines,requires,server,set,sink,source,
        subcomponents,subprogram,system,thread,to,true,type,units,value},
morecomment=[l]{--}}

% Layout for listings

\lstset{language=aadl,
        basicstyle=\scriptsize\sffamily,
        aboveskip=.1cm, % \smallskipamount, % \bigskipamount,
        belowskip=.1cm, % \smallskipamount, % \bigskipamount,
        abovecaptionskip=.1cm, % \smallskipamount, % \medskipamount,
        belowcaptionskip=.1cm, % \smallskipamount, % \bigskipamount,
        xleftmargin=.0cm,
        captionpos=b,
        tabsize=3}


\lstdefinelanguage{real}
{morekeywords={cardinal,is\_subcomponent\_of,is\_bound\_to,sum,foreach,theorem,check,end,
               memory\_set, processor\_set,min,max,
               process\_set,is\_provided\_class,is\_connected\_to,
               system\_set,thread\_set,in,bus\_set,data\_set,is\_accessed\_by,connection\_set,
               virtual\_processor\_set,Cardinal,virtual\_bus\_set,property\_exists},
morecomment=[l]{--}}


% Layout for listings

\lstset{language=aadl,
        basicstyle=\scriptsize\sffamily,
        aboveskip=.1cm, % \smallskipamount, % \bigskipamount,
        belowskip=.1cm, % \smallskipamount, % \bigskipamount,
        abovecaptionskip=.1cm, % \smallskipamount, % \medskipamount,
        belowcaptionskip=.1cm, % \smallskipamount, % \bigskipamount,
        xleftmargin=.0cm,
        captionpos=b,
        tabsize=3}


\newcommand{\bi}{\begin{itemize}}
\newcommand{\ei}{\end{itemize}}

\newcommand{\be}{\begin{enumerate}}
\newcommand{\ee}{\end{enumerate}}

\newcommand{\tbf}[1]{\textbf{#1}}
\newcommand{\tit}[1]{\textit{#1}}
\newcommand{\ttt}[1]{\texttt{#1}}
\newcommand{\tsc}[1]{\textsc{#1}}
\newcommand{\tsl}[1]{\textsl{#1}}



\newcommand{\Concept}[1]{#1\xspace}

\newcommand{\vmware}{\textsc{VMWare}\copyright\xspace}
\newcommand{\vmwareplayer}{\textsc{VMWare Player}\copyright\xspace}
\newcommand{\aadl}{\Concept{AADL}}
\newcommand{\ada}{\Concept{Ada}}
\newcommand{\arinc}{\Concept{ARINC653}}
\newcommand{\belllapadula}{\Concept{Bell-Lapadula}}
\newcommand{\biba}{\Concept{Biba}}
\newcommand{\taste}{\Concept{Taste}}
\newcommand{\cheddar}{\Concept{Cheddar}}
\newcommand{\chineesewall}{\Concept{Chineese Wall}}
\newcommand{\eclipse}{\Concept{Eclipse}}
\newcommand{\libpok}{\Concept{libpok}}
\newcommand{\marte}{\Concept{MARTE}}
\newcommand{\mils}{\Concept{MILS}}
\newcommand{\myccmhi}{\Concept{MyCCM-HI}}
\newcommand{\ocarina}{\Concept{Ocarina}}
\newcommand{\leon}{\Concept{LEON}}
\newcommand{\gnatforleon}{\Concept{Gnatforleon}}
\newcommand{\ocl}{\Concept{OCL}}
\newcommand{\osate}{\Concept{OSATE}}
\newcommand{\pok}{\Concept{POK}}
\newcommand{\posix}{\Concept{POSIX}}
\newcommand{\polyorbhi}{\Concept{PolyORB-HI}}
\newcommand{\powerpc}{\Concept{PowerPC}}
\newcommand{\qemu}{\Concept{QEMU}}
\newcommand{\real}{\Concept{REAL}}
\newcommand{\rtems}{\Concept{RTEMS}}
\newcommand{\sparc}{\Concept{Sparc}}
\newcommand{\spoq}{\Concept{SPOQ}}
\newcommand{\standardcc}{\Concept{Critères Communs}}
\newcommand{\standarddo}{\Concept{DO178B}}
\newcommand{\sysml}{\Concept{SysML}}
\newcommand{\uml}{\Concept{UML}}
\newcommand{\xml}{\Concept{XML}}
\newcommand{\xcov}{\Concept{xcov}}

\newcommand{\onehalffig}[3]{%
  \begin{figure}[htbp]
    \centerline{\epsfig{file=#1.pdf,width=.50\textwidth}}
    \caption{#2}
    \label{#3}
  \end{figure}
}

\newcommand{\onemedfig}[3]{%
  \begin{figure}[htbp]
    \centerline{\epsfig{file=#1.pdf,width=.75\textwidth}}
    \caption{#2}
    \label{#3}
  \end{figure}
}


\newcommand{\onefullfig}[3]{%
  \begin{figure}[htbp]
    \centerline{\epsfig{file=#1.pdf,width=.90\textwidth}}
    \caption{#2}
    \label{#3}
  \end{figure}
}

\newcommand{\onehugefig}[3]{%
  \begin{figure}[htbp]
    \centerline{\epsfig{file=#1.pdf,width=1\textwidth}}
    \caption{#2}
    \label{#3}
  \end{figure}
}


\newcommand{\onesmallfig}[3]{%
  \begin{figure}[htbp]
    \centerline{\epsfig{file=#1.pdf,width=.40\textwidth}}
    \caption{#2}
    \label{#3}
  \end{figure}
}


\newcommand{\onetinyfig}[3]{%
  \begin{figure}[htbp]
    \centerline{\epsfig{file=#1.pdf,width=.30\textwidth}}
    \caption{#2}
    \label{#3}
  \end{figure}
}



\title{
\centerline{\epsfig{file=imgs/assert-logo.pdf,width=.75\textwidth}}
Implementation of device drivers and bus protocols}

\author{Julien Delange}

\begin{document}

\maketitle

\newpage

\tableofcontents

\newpage

\section{Rationale}
PolyORB-HI-C is able to send and receive data from/to the network using socket
interfaces. However, we want to use other devices to exchange data 
(such as SpaceWire or serial buses) and specify how data are exchanged
(protocols used to send/receive data).

For that, we need to change \polyorbhi functions and code generation patterns.
This document presents problems and potential solutions.


\section{Problem}

   \subsection{Calls to send/receive data}
   Figure \ref{figure:actual-calls} illustrates how data are exchanged in a
   system generated from \aadl. Generated entities are depicted in red whereas
   static code (from \polyorbhi) is shown in blue. We also detail how messages
   are stored : \texttt{request} (R) or \texttt{message} (M).

   Request structure is internal to \polyorbhi. It is a structure that contains
   the variables and the port that should receive this variable. A message is a
   1024 bytes buffer where data is written.

   In this picture, we saw several problems :
   \begin{enumerate}
      \item
         Users can not tune the way messages are sent (functions used to send
         messages over the network)
      \item
         Content of messages is fixed and length is fixed by \polyorbhi.
   \end{enumerate}

   \onefullfig{imgs/actual-calls}
   {Calls to send/receive data accross a system generated from \aadl}
   {figure:actual-calls}

\section{Objectives}
The objectives are the following:
\begin{enumerate}
   \item
      \textbf{Modeling side:} describe device drivers and protocols in a
      consistent way. The output of this step is to define how we model data
      exchange between several nodes with \aadl models. In particular, we should
      rely on \texttt{device}, \texttt{bus}, \texttt{connections} and maybe
      \texttt{virtual bus} \aadl entities.
   \item
      \textbf{Code generation side:} generate calls to device drivers
      send/receive functions and give the ability to use different protocols. In
      particular, the static part of \polyorbhi and the dynamic part from the
      \aadl model must be interfaced in a way that we could
      marshall/unmarshall/send/receive data on various hardware and platforms.
\end{enumerate}

\section{Proposal}
   \subsection{Modeling}
   We propose the following modeling patterns:
   \begin{itemize}
      \item
         \texttt{Device} components model the API to send data. For TCP/IP, it is
         \texttt{send()} and \texttt{receive()} functions. You can add some
         properties to contrain the device driver. A device must be bound to a
         \texttt{processor} or \texttt{virtual processor} component.
      \item
         The internal implementation of a device driver is specified using the
         \texttt{Device\_Driver} property associated with the \texttt{device}
         component. This property contains an \texttt{abstract} component
         classifier with all required components that implement the driver.
      \item
         Each device that accesses to a bus must declare a \texttt{bus
         access} (in \texttt{features}) so that we can connect the device to its bus. If several buses
         of the same type are used in the model, we can precisely know on which
         one it operates.
      \item
         \texttt{Bus} components represent the medium that transport the data. It does
         not specify any protocol.
      \item
         \texttt{Virtual bus} components model a protocol.
      \item
         The association between a \texttt{virtual bus} and a \texttt{bus}
         component is specified by components containment : the \texttt{bus}
         contains one or several \texttt{virtual bus} to describe which protocol
         it can transport.
      \item
         A \texttt{device} component specifies which protocols it supports by
         specifying the \texttt{Provided\_Virtual\_Bus\_Class} property.
      \item
         Each connection must be bound with the 
         \texttt{Actual\_Connection\_Binding} property with :
         \begin{enumerate}
            \item
               A \texttt{virtual bus} contained in a \texttt{bus}
            \item
               The \texttt{device} that send the data on the current node.
         \end{enumerate}
   \end{itemize}


   \subsection{Model example}
   The following model gives an illustration on how we can model connections
   between two processes that exchanges data using several buses. The model is
   not complete but it gives an overview of the modeling patterns.

   \begin{lstlisting}[language=aadl]
   bus implementation eth_network.i
   subcomponents
      tcp : virtual bus tcp_ip.i;
   end eth_network.i;

   bus implementation uart_bus.i
   subcomponents
      dummy : virtual bus dummy.i;
   end uart_bus.i;

   device implementation uart.i
   properties
      Provided_Virtual_Bus_Class => (classifier (virtual bus dummy.i));
   end uart.i;

   device implementation eth.i
   properties
      Provided_Virtual_Bus_Class => (classifier (virtual bus tcp_ip.i));
   end eth.i;


   system implementation my_system.i
   subcomponents
      process_sender       : device sender.i;
      process_receiver     : device receiver.i;

      device_uart_sender   : device uart.i;
      device_uart_receive  : device uart.i;

      device_eth_sender    : device eth.i;
      device_eth_receive   : device eth.i;

      cpu_sender           : processor powerpc.i;
      cpu_receive          : processor x86.i;

      eth                  : bus eth_network.i;
      uart                 : bus uart_bus.i;
   connections
      bus access eth -> device_eth_sender.device_access;
      bus access eth -> device_eth_receiver.device_access;

      bus access uart -> device_uart_sender.device_access;
      bus access uart -> device_uart_receiver.device_access;

      process_sender.source1 -> process_receiver.sink1
         {Actual_Connection_Binding (reference (eth.tcp))};
      process_sender.source2 -> process_receiver.sink2
         {Actual_Connection_Binding (reference (uart.dummy))};
   properties
      Actual_Processor_Binding => (reference (cpu_sender))
         applies to device_uart_sender;
      Actual_Processor_Binding => (reference (cpu_sender))
         applies to device_eth_sender;
      Actual_Processor_Binding => (reference (cpu_sender))
         applies to process_sender;

      Actual_Processor_Binding => (reference (cpu_receiver))
         applies to device_uart_receiver;
      Actual_Processor_Binding => (reference (cpu_receiver))
         applies to device_eth_receiver;
      Actual_Processor_Binding => (reference (cpu_receiver))
         applies to process_receiver;
   end my_system.i;
   \end{lstlisting}


   \subsection{Potential validation rules}
   For a validation point of view, we can then check that :
   \begin{itemize}
      \item
         Protocols of each \texttt{bus} is supported by at least one
         \texttt{device} connected to it.
      \item
         Devices are able to support each connection (they support the
         appropriate protocol).
      \item
         Process that exchange data through an \aadl connection can actually send 
         it through devices that shares the same bus (use \texttt{access}
         connections to do this analysis/validation).
   \end{itemize}


   \subsection{Code Generation}
   Figure \ref{figure:proposed-calls} illustrates the proposed call sequence. In
   particular, it shows how messages are exchanged. We see the main differences:
   \begin{itemize}
      \item
         Generated code handle only \texttt{request} type. It does not longer
         care about messages.
      \item
         Messages are completly handled by device drivers so that they handle
         their own size, marshalling, \ldots
      \item
         We introduce a delivery service. Depending on how device communicate
         and their associated communication patterns (1-to-N, N-to-N, 1-to-1, 
         \ldots), it can use a different delivery service. We introduce two
         delivery methods :
         \begin{enumerate}
            \item
               One that deliver a request to all connected entities (such as a
               broadcast). In this delivery method, the message is sent, no
               matter how many recipients the device has. It is a convenient way
               to send message for a N-to-1 or 1-to-1 communication pattern.
            \item
               One that deliver a request to connected entities only. This is
               the old \texttt{main\_deliver} method.
         \end{enumerate}
   \end{itemize}

   \onefullfig{imgs/proposed-calls}
   {Proposal to change the call sequence in generated applications}
   {figure:proposed-calls}


      \subsubsection{Additional work in \ocarina}
      The following information should be added in the generated code to support
      different protocols/device.

      \begin{itemize}
         \item
            Add an endianess configuration directive so that device drivers know on
            which architecture they are running (useful for marshall/unmarshall functions).
            We can do that with a macro such as \texttt{POLYORB\_HI\_BIGENDIAN}
            (set to 0 or 1, depending on the \texttt{Execution\_Platform} property
            of the \aadl \texttt{processor} component).
      \end{itemize}

\end{document}


%% bare_conf.tex
%% V1.3
%% 2007/01/11
%% by Michael Shell
%% See:
%% http://www.michaelshell.org/
%% for current contact information.
%%
%% This is a skeleton file demonstrating the use of IEEEtran.cls
%% (requires IEEEtran.cls version 1.7 or later) with an IEEE conference paper.
%%
%% Support sites:
%% http://www.michaelshell.org/tex/ieeetran/
%% http://www.ctan.org/tex-archive/macros/latex/contrib/IEEEtran/
%% and
%% http://www.ieee.org/

%%*************************************************************************
%% Legal Notice:
%% This code is offered as-is without any warranty either expressed or
%% implied; without even the implied warranty of MERCHANTABILITY or
%% FITNESS FOR A PARTICULAR PURPOSE! 
%% User assumes all risk.
%% In no event shall IEEE or any contributor to this code be liable for
%% any damages or losses, including, but not limited to, incidental,
%% consequential, or any other damages, resulting from the use or misuse
%% of any information contained here.
%%
%% All comments are the opinions of their respective authors and are not
%% necessarily endorsed by the IEEE.
%%
%% This work is distributed under the LaTeX Project Public License (LPPL)
%% ( http://www.latex-project.org/ ) version 1.3, and may be freely used,
%% distributed and modified. A copy of the LPPL, version 1.3, is included
%% in the base LaTeX documentation of all distributions of LaTeX released
%% 2003/12/01 or later.
%% Retain all contribution notices and credits.
%% ** Modified files should be clearly indicated as such, including  **
%% ** renaming them and changing author support contact information. **
%%
%% File list of work: IEEEtran.cls, IEEEtran_HOWTO.pdf, bare_adv.tex,
%%                    bare_conf.tex, bare_jrnl.tex, bare_jrnl_compsoc.tex
%%*************************************************************************

% *** Authors should verify (and, if needed, correct) their LaTeX system  ***
% *** with the testflow diagnostic prior to trusting their LaTeX platform ***
% *** with production work. IEEE's font choices can trigger bugs that do  ***
% *** not appear when using other class files.                            ***
% The testflow support page is at:
% http://www.michaelshell.org/tex/testflow/



% Note that the a4paper option is mainly intended so that authors in
% countries using A4 can easily print to A4 and see how their papers will
% look in print - the typesetting of the document will not typically be
% affected with changes in paper size (but the bottom and side margins will).
% Use the testflow package mentioned above to verify correct handling of
% both paper sizes by the user's LaTeX system.
%
% Also note that the "draftcls" or "draftclsnofoot", not "draft", option
% should be used if it is desired that the figures are to be displayed in
% draft mode.
%
\documentclass[10pt, conference, compsocconf]{IEEEtran}
\usepackage{hyperref}
\usepackage{acronym}
\usepackage{epsfig}
\usepackage{listings}

\lstdefinelanguage{asn}
{morekeywords={definitions,automatic,tags,default,integer},
morecomment=[l]{--}}

\newcounter{figcounter}
\def\epsin #1#2#3#4{
\refstepcounter{figcounter} \label{#3}
\[
\mbox{
  \epsfxsize=#2mm
  \epsffile{#1.eps}
}
\]
%\vspace{0mm}
\begin{center}
  \parbox{7cm}{{\bf FIGURE \arabic{figcounter}:}\quad {\it #4 } } \\
\end{center}
}

% Add the compsocconf option for Computer Society conferences.
%
% If IEEEtran.cls has not been installed into the LaTeX system files,
% manually specify the path to it like:
% \documentclass[conference]{../sty/IEEEtran}





% Some very useful LaTeX packages include:
% (uncomment the ones you want to load)


% *** MISC UTILITY PACKAGES ***
%
%\usepackage{ifpdf}
% Heiko Oberdiek's ifpdf.sty is very useful if you need conditional
% compilation based on whether the output is pdf or dvi.
% usage:
% \ifpdf
%   % pdf code
% \else
%   % dvi code
% \fi
% The latest version of ifpdf.sty can be obtained from:
% http://www.ctan.org/tex-archive/macros/latex/contrib/oberdiek/
% Also, note that IEEEtran.cls V1.7 and later provides a builtin
% \ifCLASSINFOpdf conditional that works the same way.
% When switching from latex to pdflatex and vice-versa, the compiler may
% have to be run twice to clear warning/error messages.






% *** CITATION PACKAGES ***
%
%\usepackage{cite}
% cite.sty was written by Donald Arseneau
% V1.6 and later of IEEEtran pre-defines the format of the cite.sty package
% \cite{} output to follow that of IEEE. Loading the cite package will
% result in citation numbers being automatically sorted and properly
% "compressed/ranged". e.g., [1], [9], [2], [7], [5], [6] without using
% cite.sty will become [1], [2], [5]--[7], [9] using cite.sty. cite.sty's
% \cite will automatically add leading space, if needed. Use cite.sty's
% noadjust option (cite.sty V3.8 and later) if you want to turn this off.
% cite.sty is already installed on most LaTeX systems. Be sure and use
% version 4.0 (2003-05-27) and later if using hyperref.sty. cite.sty does
% not currently provide for hyperlinked citations.
% The latest version can be obtained at:
% http://www.ctan.org/tex-archive/macros/latex/contrib/cite/
% The documentation is contained in the cite.sty file itself.






% *** GRAPHICS RELATED PACKAGES ***
%
\ifCLASSINFOpdf
  % \usepackage[pdftex]{graphicx}
  % declare the path(s) where your graphic files are
  % \graphicspath{{../pdf/}{../jpeg/}}
  % and their extensions so you won't have to specify these with
  % every instance of \includegraphics
  % \DeclareGraphicsExtensions{.pdf,.jpeg,.png}
\else
  % or other class option (dvipsone, dvipdf, if not using dvips). graphicx
  % will default to the driver specified in the system graphics.cfg if no
  % driver is specified.
  % \usepackage[dvips]{graphicx}
  % declare the path(s) where your graphic files are
  % \graphicspath{{../eps/}}
  % and their extensions so you won't have to specify these with
  % every instance of \includegraphics
  % \DeclareGraphicsExtensions{.eps}
\fi
% graphicx was written by David Carlisle and Sebastian Rahtz. It is
% required if you want graphics, photos, etc. graphicx.sty is already
% installed on most LaTeX systems. The latest version and documentation can
% be obtained at: 
% http://www.ctan.org/tex-archive/macros/latex/required/graphics/
% Another good source of documentation is "Using Imported Graphics in
% LaTeX2e" by Keith Reckdahl which can be found as epslatex.ps or
% epslatex.pdf at: http://www.ctan.org/tex-archive/info/
%
% latex, and pdflatex in dvi mode, support graphics in encapsulated
% postscript (.eps) format. pdflatex in pdf mode supports graphics
% in .pdf, .jpeg, .png and .mps (metapost) formats. Users should ensure
% that all non-photo figures use a vector format (.eps, .pdf, .mps) and
% not a bitmapped formats (.jpeg, .png). IEEE frowns on bitmapped formats
% which can result in "jaggedy"/blurry rendering of lines and letters as
% well as large increases in file sizes.
%
% You can find documentation about the pdfTeX application at:
% http://www.tug.org/applications/pdftex





% *** MATH PACKAGES ***
%
%\usepackage[cmex10]{amsmath}
% A popular package from the American Mathematical Society that provides
% many useful and powerful commands for dealing with mathematics. If using
% it, be sure to load this package with the cmex10 option to ensure that
% only type 1 fonts will utilized at all point sizes. Without this option,
% it is possible that some math symbols, particularly those within
% footnotes, will be rendered in bitmap form which will result in a
% document that can not be IEEE Xplore compliant!
%
% Also, note that the amsmath package sets \interdisplaylinepenalty to 10000
% thus preventing page breaks from occurring within multiline equations. Use:
%\interdisplaylinepenalty=2500
% after loading amsmath to restore such page breaks as IEEEtran.cls normally
% does. amsmath.sty is already installed on most LaTeX systems. The latest
% version and documentation can be obtained at:
% http://www.ctan.org/tex-archive/macros/latex/required/amslatex/math/





% *** SPECIALIZED LIST PACKAGES ***
%
%\usepackage{algorithmic}
% algorithmic.sty was written by Peter Williams and Rogerio Brito.
% This package provides an algorithmic environment fo describing algorithms.
% You can use the algorithmic environment in-text or within a figure
% environment to provide for a floating algorithm. Do NOT use the algorithm
% floating environment provided by algorithm.sty (by the same authors) or
% algorithm2e.sty (by Christophe Fiorio) as IEEE does not use dedicated
% algorithm float types and packages that provide these will not provide
% correct IEEE style captions. The latest version and documentation of
% algorithmic.sty can be obtained at:
% http://www.ctan.org/tex-archive/macros/latex/contrib/algorithms/
% There is also a support site at:
% http://algorithms.berlios.de/index.html
% Also of interest may be the (relatively newer and more customizable)
% algorithmicx.sty package by Szasz Janos:
% http://www.ctan.org/tex-archive/macros/latex/contrib/algorithmicx/




% *** ALIGNMENT PACKAGES ***
%
%\usepackage{array}
% Frank Mittelbach's and David Carlisle's array.sty patches and improves
% the standard LaTeX2e array and tabular environments to provide better
% appearance and additional user controls. As the default LaTeX2e table
% generation code is lacking to the point of almost being broken with
% respect to the quality of the end results, all users are strongly
% advised to use an enhanced (at the very least that provided by array.sty)
% set of table tools. array.sty is already installed on most systems. The
% latest version and documentation can be obtained at:
% http://www.ctan.org/tex-archive/macros/latex/required/tools/


%\usepackage{mdwmath}
%\usepackage{mdwtab}
% Also highly recommended is Mark Wooding's extremely powerful MDW tools,
% especially mdwmath.sty and mdwtab.sty which are used to format equations
% and tables, respectively. The MDWtools set is already installed on most
% LaTeX systems. The lastest version and documentation is available at:
% http://www.ctan.org/tex-archive/macros/latex/contrib/mdwtools/


% IEEEtran contains the IEEEeqnarray family of commands that can be used to
% generate multiline equations as well as matrices, tables, etc., of high
% quality.


%\usepackage{eqparbox}
% Also of notable interest is Scott Pakin's eqparbox package for creating
% (automatically sized) equal width boxes - aka "natural width parboxes".
% Available at:
% http://www.ctan.org/tex-archive/macros/latex/contrib/eqparbox/





% *** SUBFIGURE PACKAGES ***
%\usepackage[tight,footnotesize]{subfigure}
% subfigure.sty was written by Steven Douglas Cochran. This package makes it
% easy to put subfigures in your figures. e.g., "Figure 1a and 1b". For IEEE
% work, it is a good idea to load it with the tight package option to reduce
% the amount of white space around the subfigures. subfigure.sty is already
% installed on most LaTeX systems. The latest version and documentation can
% be obtained at:
% http://www.ctan.org/tex-archive/obsolete/macros/latex/contrib/subfigure/
% subfigure.sty has been superceeded by subfig.sty.



%\usepackage[caption=false]{caption}
%\usepackage[font=footnotesize]{subfig}
% subfig.sty, also written by Steven Douglas Cochran, is the modern
% replacement for subfigure.sty. However, subfig.sty requires and
% automatically loads Axel Sommerfeldt's caption.sty which will override
% IEEEtran.cls handling of captions and this will result in nonIEEE style
% figure/table captions. To prevent this problem, be sure and preload
% caption.sty with its "caption=false" package option. This is will preserve
% IEEEtran.cls handing of captions. Version 1.3 (2005/06/28) and later 
% (recommended due to many improvements over 1.2) of subfig.sty supports
% the caption=false option directly:
%\usepackage[caption=false,font=footnotesize]{subfig}
%
% The latest version and documentation can be obtained at:
% http://www.ctan.org/tex-archive/macros/latex/contrib/subfig/
% The latest version and documentation of caption.sty can be obtained at:
% http://www.ctan.org/tex-archive/macros/latex/contrib/caption/




% *** FLOAT PACKAGES ***
%
%\usepackage{fixltx2e}
% fixltx2e, the successor to the earlier fix2col.sty, was written by
% Frank Mittelbach and David Carlisle. This package corrects a few problems
% in the LaTeX2e kernel, the most notable of which is that in current
% LaTeX2e releases, the ordering of single and double column floats is not
% guaranteed to be preserved. Thus, an unpatched LaTeX2e can allow a
% single column figure to be placed prior to an earlier double column
% figure. The latest version and documentation can be found at:
% http://www.ctan.org/tex-archive/macros/latex/base/



%\usepackage{stfloats}
% stfloats.sty was written by Sigitas Tolusis. This package gives LaTeX2e
% the ability to do double column floats at the bottom of the page as well
% as the top. (e.g., "\begin{figure*}[!b]" is not normally possible in
% LaTeX2e). It also provides a command:
%\fnbelowfloat
% to enable the placement of footnotes below bottom floats (the standard
% LaTeX2e kernel puts them above bottom floats). This is an invasive package
% which rewrites many portions of the LaTeX2e float routines. It may not work
% with other packages that modify the LaTeX2e float routines. The latest
% version and documentation can be obtained at:
% http://www.ctan.org/tex-archive/macros/latex/contrib/sttools/
% Documentation is contained in the stfloats.sty comments as well as in the
% presfull.pdf file. Do not use the stfloats baselinefloat ability as IEEE
% does not allow \baselineskip to stretch. Authors submitting work to the
% IEEE should note that IEEE rarely uses double column equations and
% that authors should try to avoid such use. Do not be tempted to use the
% cuted.sty or midfloat.sty packages (also by Sigitas Tolusis) as IEEE does
% not format its papers in such ways.





% *** PDF, URL AND HYPERLINK PACKAGES ***
%
%\usepackage{url}
% url.sty was written by Donald Arseneau. It provides better support for
% handling and breaking URLs. url.sty is already installed on most LaTeX
% systems. The latest version can be obtained at:
% http://www.ctan.org/tex-archive/macros/latex/contrib/misc/
% Read the url.sty source comments for usage information. Basically,
% \url{my_url_here}.





% *** Do not adjust lengths that control margins, column widths, etc. ***
% *** Do not use packages that alter fonts (such as pslatex).         ***
% There should be no need to do such things with IEEEtran.cls V1.6 and later.
% (Unless specifically asked to do so by the journal or conference you plan
% to submit to, of course. )


% correct bad hyphenation here
\hyphenation{op-tical net-works semi-conduc-tor}


\begin{document}
%
% paper title
% can use linebreaks \\ within to get better formatting as desired
\title{Towards a Model-Driven Engineering Software Development Framework}


% author names and affiliations
% use a multiple column layout for up to two different
% affiliations

\author{\IEEEauthorblockN{Maxime Perrotin, Julien Delange and Samir Bennani}
\IEEEauthorblockA{European Space Agency, TEC-SWE\\
Keplerlaan 1\\
2201AZ Noordwijk, The Netherlands\\
Email: firstname.lastname@esa.int}
}

% conference papers do not typically use \thanks and this command
% is locked out in conference mode. If really needed, such as for
% the acknowledgment of grants, issue a \IEEEoverridecommandlockouts
% after \documentclass

% for over three affiliations, or if they all won't fit within the width
% of the page, use this alternative format:
% 
%\author{\IEEEauthorblockN{Michael Shell\IEEEauthorrefmark{1},
%Homer Simpson\IEEEauthorrefmark{2},
%James Kirk\IEEEauthorrefmark{3}, 
%Montgomery Scott\IEEEauthorrefmark{3} and
%Eldon Tyrell\IEEEauthorrefmark{4}}
%\IEEEauthorblockA{\IEEEauthorrefmark{1}School of Electrical and Computer Engineering\\
%Georgia Institute of Technology,
%Atlanta, Georgia 30332--0250\\ Email: see http://www.michaelshell.org/contact.html}
%\IEEEauthorblockA{\IEEEauthorrefmark{2}Twentieth Century Fox, Springfield, USA\\
%Email: homer@thesimpsons.com}
%\IEEEauthorblockA{\IEEEauthorrefmark{3}Starfleet Academy, San Francisco, California 96678-2391\\
%Telephone: (800) 555--1212, Fax: (888) 555--1212}
%\IEEEauthorblockA{\IEEEauthorrefmark{4}Tyrell Inc., 123 Replicant Street, Los Angeles, California 90210--4321}}




% use for special paper notices
%\IEEEspecialpapernotice{(Invited Paper)}




% make the title area
\maketitle


\begin{abstract}
Design and Implementation of Safety-Critical Systems is becoming very difficult
becauses it involves many requirements coming from different engineering domains. Due
to the increase of complexity, software of such systems can no longer be produced
with traditional methods, which show their limit over time. In that context,
new development approaches have to be introduced to 
avoid actual development traps and pitfalls. Among them,
the Model-Driven Engineering approach consists at representing
system artifacts with models and auto-generate the code by refining
them from high-level concepts down to the code. However, as for every
new approach, it also brings new problems such as requirements consistency
among the different notations (models) as well as integration
issues (for example, making sure that implementation code 
from different models will behave correctly when merged on a single execution
platform).

This article presents our experience for integrating Guidance
and Navigation Control (GNC) algorithms designed with Application Models
(Simulink) with Architecture Models (AADL). The process
relies on code generator for both models and integrate it on a typical execution
platform.
In particular, we focus on the challenges of the integration, illustrating
the practical problems we faced for producing a space system
using a Model-Driven Engineering Approach.

\end{abstract}

\begin{IEEEkeywords}
AADL, TASTE, Simulink, MDE
\end{IEEEkeywords}


% For peer review papers, you can put extra information on the cover
% page as needed:
% \ifCLASSOPTIONpeerreview
% \begin{center} \bfseries EDICS Category: 3-BBND \end{center}
% \fi
%
% For peerreview papers, this IEEEtran command inserts a page break and
% creates the second title. It will be ignored for other modes.
\IEEEpeerreviewmaketitle



\section{Introduction}
% no \IEEEPARstart

\subsection{Context}
Safety-critical systems are getting more complex, collocating more
functions on the same computing platform. As a consequence, their design
becomes more complicated, leading to a long and potentially painful
design process. Designers have to take into account requirements
coming from different domains and specified with heterogeneous formalisms.

For that reason, producing the system using traditional methods is
no longer feasible : checking impact between all requirements
disseminated across different specifications and notations is 
impossible, especially when the process is not automated. For that
reason, new approaches must be designed. In our context, the Model-Driven
Engineering (MDE) approach aims at separating system concerns in models,
let engineers focus on their part of the system while tools
automate the integration and ensure consistencies between modelling
artifacts.


\subsection{Problem}
One key aspect of the MDE methodology is
the separation of concerns: each engineer focuses on designing
and implementing his part of the system dedicated to his domain
while specific tools process each implementation artifact, ensure 
their integration and preserve a semantic consistency between 
notations. 

Despite having a clear separation between each domain,
several problems remain when implementing a full MDE
approach. First of all, because of the use of different
notations, some requirements are sometimes captured twice in
different models using heterogeneous notations. Then, tools are expected
to ensure their consistency but engineers are also requested
not to break them when manipulating models. Another issue
is about simulation and implementation correctness: even
when a system was intensively tested using simulation functions, its
integration as an implementation code can generate a lot of errors.
Most of the time, this is due to the heterogeneous nature of the
execution platform, whose environment is different from the simulation.
Thus, having a dedicated process that ensures a smooth integration 
of system functions by enforcing their behavior correctness is a must.

In that context, to improve MDE approaches, it is worthwhile 
to identify all traps and pitfalls in order to strengthen the overall
approach. Outcome of such investigations would make tools more 
resilient to potential integration errors, the type of issue which is 
typically discovered just before completion of
a project, when few resources are available 

\subsection{Outline}
The remainder of this paper is structured in two main
sections. The first one is an overview of the tools we use:
the TASTE toolset (Architecture modelling~\cite{bass2003software}), 
 and Simulink~\cite{EE6393A162144B9BBB6B44D9BC0C4742} (Software modelling) and their 
integration. Second part of this paper presents
our latest experiments to design a spacecraft 
system using our MDE toolset by designing 
GNC algorithms on top of a distributed architecture
designed with TASTE. We provide a feedback about traditional
traps and pitfalls of such integration, leading to
an open discussion about potential improvements of the whole MDE development
process.



% You must have at least 2 lines in the paragraph with the drop letter
% (should never be an issue)

\section{Background}

   \subsection{Simulink}
   Simulink consists of a graphical modelling language and a set of tools for designing
   software (an example is shown in figure \ref{fig:simulink-model}). It focuses on the definition of functional
   concerns and is mostly used to design algorithms in specific engineering
   domain (power control, navigation, etc.).

   Simulink is a well-known and established tool, providing
   convenient notation to abstract engineering concepts with
   a user-friendly simulation interface. Thus, it is a very efficient
   tool for engineers to prototype, design
   and implement their part of the system. In the context of the space
   industry, it is used in many engineering domains, from mechanical to robotics.
   In the present case-study, it was used to produce the GNC
   algorithms of a launcher.

   \subsection{TASTE}
   TASTE~\cite{delange11sdl} is a project developed, maintained and supported by the European Space
   Agency. It aims at providing a MDE toolset for the production
   of safety-critical systems. For that purpose, it defines the system using
   three views:
   \begin{enumerate}
      \item
         The \textbf{Data View} (DaV) specifies data types and encoding functions
         used to communicate between system components using the ASN.1~\cite{dubuisson1999asn} language
         (see listing \ref{listing:data-view} for an example). 
         It corresponds to the external interfaces of the system, as specified in the ICD that specified
         all interfaces between the sub-systems (with their types, properties,
         etc.).
      \item
         The \textbf{Interface View} (IV) enumerates system functions (what the
         system is doing), their requirements and constraints (timing, data protection, etc.)
         and interactions among then (communication channels) using the AADL~\cite{aadl} language. Function connections
         reference the \textbf{Data View} to specify types being used.
         As this model remains descriptive (a graphical sample is shown in
         figure \ref{fig:taste-model-iv}), it has to be associated
         with code that implements functions.
      \item
         \textbf{Deployment View} (DeV) defines the execution platform 
         (processors, buses, etc.) of the system, its configuration 
         as well as deployment of functions (from the \textbf{Interface View}) into it.
         It also uses the AADL~\cite{aadl} for that purpose, an example is shown
         in figure \ref{fig:taste-model-dv}.
   \end{enumerate}

   These models are processed with their functional code to
   automatically produce system implementation (as shown in figure \ref{fig:taste-process})
   through the following steps:
   \begin{itemize}
      \item
         The \textbf{Data View} is translated into declarations (data types, functions) in the target code
         (C, Ada, etc.) so that we can use the same interfaces with different
         languages.
      \item
         The \textbf{Interface} and \textbf{Deployment Views} are processed
         to create an architecture code that supports the execution
         of the functional code. This aims at creating and configuring resources
         (tasks, mutexes, etc.) that reflect requirements specified in the
         interface view (period, deadline, etc.).
      \item
         The \textbf{Integration Process} compiles the functional code from the user
         with the architecture code generated from models, producing the program
         to be deployed on the execution target.
         This code is also automatically tailored to the target operating
         system. As for now, our toolchain supports several architecture
         (x86, SPARC, etc.) and different Operating System for safety-critical
         systems such as Linux or RTEMS~\cite{RTEMS}.
   \end{itemize}

   \epsin{imgs/taste-process}{80}{fig:taste-process}{TASTE development process}


   \subsection{Integration of Software Models into Architecture Models}
   Our process (figure \ref{fig:taste-process}) automatically deploys
   application code (written by domain-specific users, such as
   electrical/mechanical engineers) on top of architecture code.
   The former can be designed
   using either regular (C, Ada) or modelling (Simulink, SDL, etc.) languages. Our
   toolchain automatically generates glue code that connects functional
   blocks, enabling communication between code written with different languages
   and executed on heterogeneous architectures.
   
   However, when integrating application models (such as Simulink), algorithms requirements 
   must comply with the architecture constraints (timing requirements of the
   \textbf{Interface View}, interfaces definition of the \textbf{Data View},
   etc.). Using a traditional, manual integration, no check is performed, lack of compliance between models
   is discovered either after integration (at best), or during execution (at
   worst). The following section contains the description of a complete case study we conducted
   in order to experiment and validate the use of a model-based approach with the TASTE tools.


\section{Case-Study and feedback}

   \subsection{Overview}

   In order to validate our approach and tools, we have built a system to support the validation
   of the navigation algorithms (GNC) of a onboard launcher software. In that context, we have developed
   (or reused) a large set of Simulink models that represent the environment of the software: the sensors
   and actuators on one side, and the flight dynamics on the other side. This way we have the means
   to generate realistic data at runtime to feed the control laws and run in closed loop.

   On the other side, we have the flight code of the control laws in Ada language. The challenge is to
   put both pieces together, and make them run on different platform: first natively on Linux to test
   the integration, and later on a mixed platforms (x86/Linux for the environment models, and Sparc/Leon
   for the control law running in real-time). At runtime, we want to observe data and plot the control laws'
   main parameters. TASTE provide the means to achieve these goals.

   \subsection{Application modelling}

   \epsin{imgs/simulink-model}{80}{fig:simulink-model}{Simulink model of our case-study}

   Modelling the application with TASTE was done in three steps
   \begin{enumerate}
      \item
         specify the data types in ASN.1
         to describe the messages exchanged between our functions (listing \ref{listing:data-view})
      \item
         capture the logical architecture
         of the system (figure \ref{fig:taste-model-iv}) into an Interface View that references 
         the Simulink model (shown in {fig:simulink-model}). 
      \item
         model the deployment of the application, by mapping the functions onto hardware components, and connect them
         with buses (figure \ref{fig:taste-model-dv}).
   \end{enumerate}

   From these models, TASTE generate \textit{skeletons}, that consist in empty code blocks that would contain
   the application (Simulink models). Once these code blocks have been filled by the user, 
   tools automatically create all the glue code that
   is necessary to implement the system (communication, etc.) without having to manually tweak the interfaces. This 
   auto-connects blocks from different implementation languages (Ada, C,
   Simulink, etc.) smoothly, without having to change the communication
   mechanisms. Using these tools gave us guarantee
   that there would be no inconsistencies in data representation for each block. This allows to start running
   simulations very quickly and make rapid prototyping of embedded application without having to tweak application code.
  
   \lstinputlisting[caption={Data View of our case-study},label=listing:data-view,language=asn,basicstyle=\footnotesize,frame=single,numberstyle=\tiny,captionpos=b]{data-view.asn}

   \epsin{imgs/taste-model-iv}{80}{fig:taste-model-iv}{Interface View of our case-study}


   \subsection{Feedback}

   The experience gives us the ability to produce a large scale launcher simulator that we are now
   running to cross-check the control laws of our launchers. Until now, we did most 
   of these simulations entirely within the Matlab/Simulink environment. 
   It is also a very powerful and effective approach; the drawback
   is that the execution is not representative of real targets, and that the integration of the code later on
   in the real onboard software can become much more difficult when done at a later stage. Using automated tools
   to make the models and code integration from the very early stages of the development brings significant added
   value and makes things simpler to integrate for non-software people.

   However, we experienced two major issues when integrating Simulink code on the real
   target. First of all, the generated code from application models is not
   fully consistent with the models specification, leading to errors that were not
   seen during simulation. This brings us to debug the generated code by hand to
   discover the problem. This should be automatically detected by the tools. Also, another
   issue was the programming skills of engineers : the toolchain required
   adaptation of existing Simulink models to the TASTE interface. This job requires some
   model refactoring to fit TASTE and Simulink models and was not easy for non
   computer-scientist engineers. As a result, we spent a lot of time explaining
   the design process to engineers so that they can use our toolchain.

\section{Conclusion}
This article presents our feedback about the issues of the deployment
of software models with architecture, particularly regarding potential
errors that are introduced during the integration phase, at the latest
phases of system production. These experiments were done in the context
of internal projects at the European Space Agency, while integrating
GNC algorithms (designed and tested usign simulation functions
from Simulink) with an execution runtime.

These issues convince us to strenghthen the overall development process
and propose new functionalities that aim at checking system compliance
between execution and simulation. For example, being able to monitor system
interfaces and check correctness between simulation and execution
would help developers but also support the overall development process,
providing artifacts required for system validation.

   \subsection{Perspectives}
   Verification between models could also be introduced earlier in the
   development process, prior to system integration. For example, it would be
   possible to check compliance between heterogeneous models before generating 
   or integrating code. This engineering effort, even if technically feasible, would
   require a static analysis of the model, which requires a huge
   maintainance effort due to the number of application languages supported by our
   toolchain.

\*section{Aknowledgment} 

The authors would like to thank the VEGA project and the TEC-ECN and TEC-SW laboratories for their support and contributions to this paper. 

\section*{Acronyms}
\begin{acronym}
\acro{ASN}{~~ Abstract Syntax Notation}
\acro{ICD}{~~ Interface Control Description}
\acro{GNC}{~~ Guidance and Navigation Control}
\acro{MDE}{~~ Model-Driven Engineering}
\acro{TASTE}{~~ The ASSERT Set of Tools for Engineering}
\end{acronym}


   \epsin{imgs/taste-model-dv}{80}{fig:taste-model-dv}{Deployment View of our case-study}


% trigger a \newpage just before the given reference
% number - used to balance the columns on the last page
% adjust value as needed - may need to be readjusted if
% the document is modified later
%\IEEEtriggeratref{8}
% The "triggered" command can be changed if desired:
%\IEEEtriggercmd{\enlargethispage{-5in}}

% references section

% can use a bibliography generated by BibTeX as a .bbl file
% BibTeX documentation can be easily obtained at:
% http://www.ctan.org/tex-archive/biblio/bibtex/contrib/doc/
% The IEEEtran BibTeX style support page is at:
% http://www.michaelshell.org/tex/ieeetran/bibtex/
\bibliographystyle{IEEEtran}
% argument is your BibTeX string definitions and bibliography database(s)
\bibliography{biblio}



% that's all folks
\end{document}


